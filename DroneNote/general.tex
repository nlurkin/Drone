In the following section all the forces will be expressed in the reference frame
of the quadcopter (the body frame) with the x and y axis along the perpendicular
arms and the z axis at the center of the model and pointing upwards as depicted
on \fig{RefFrame}. The computation is much simpler in this frame because all the
forces and torques involved are oriented along the z axis. On the other hand all
the positions, speeds and orientations are expressed in the inertial frame and
represent the motion of the body frame observed in the inertial frame. Moving
from the body frame to the inertial one can be done by using the rotation
matrices associated to the roll-pitch-yaw angles $\v{\theta} =
(\phi,\theta,\psi)$ describing the orientation of the body frame in the inertial frame.
\begin{equation}
	R_{b,i} = \matrix{
	c\phi c\psi - c\theta s\phi s\psi & -c\psi s\phi - c\phi c\theta s\psi &
	s\theta s\psi\\
	c\theta c\psi s\phi+c\phi s\psi & c\phi c\theta c\psi-s\phi s\psi &
	-c\psi s\theta\\
	s\phi s\theta & c\phi s\theta & c\theta
	}
\end{equation}
Moving from the inertial frame to the body frame is done using the inverse
matrix $R^{-1}$.

\subsection{Quaternion representation}
We will however avoid using angles whenever possible because of their intrinsic
gimbal-lock problem\todo{reference}. We will rather use the
unit quaternion\todo{reference} representation which has several advantages:
\begin{itemize}
  \item Not subject to gimbal lock
  \item More compact
  \item Easier to compose
  \item Numerically more stable and computationally less costly
\end{itemize}
The quaternion is formed of three complex coordinates x,y,z (the
vector part) and a real coordinate w: $\q{q} = w + x\mathbf{i} + y\mathbf{j} +
z\mathbf{k}$. The norm of the quaternion is defined as:
\begin{equation}
	||\q{q}|| = w^2 + x^2 + y^2 + z^2
\end{equation}
It basically represents a
single axis in 3D space around which the rotation is done. In order to represent rotations the quaternion must
always be a unit quaternion, i.e. its norm is 1.

Constructing the quaternion from the \v{\theta} angles is done with the
following operation:
\begin{equation}
	\matrix{w\\x\\y\\z} = \matrix{
		c(\phi/2)c(\theta/2)c(\psi/2)+s(\phi/2)s(\theta/2)s(\psi/2)\\
		s(\phi/2)c(\theta/2)c(\psi/2)-c(\phi/2)s(\theta/2)s(\psi/2)\\
		c(\phi/2)s(\theta/2)c(\psi/2)+s(\phi/2)c(\theta/2)s(\psi/2)\\
		c(\phi/2)c(\theta/2)s(\psi/2)-s(\phi/2)s(\theta/2)c(\psi/2)
	}
\end{equation}
 Going back to the angles is done with:
 \begin{equation}
	\matrix{\phi\\\theta\\\psi} = \matrix{
		arctan(\frac{2(wx + yz)}{1-2(x^2 + y^2)})\\
		arcsin(2(wy - zx))\\
		arctan(\frac{2(wz + xy)}{1-2(y^2y + z^2)})
	}
\end{equation}

The reverse rotation is done by taking the inverse quaternion which is equal to
its conjugate since it is a unit quaternion:
\begin{equation}
	\q{q^{-1}} = \q{q^{*}} = \matrix{w\\-x\\-y\\-z} 
\end{equation}

Composition of rotations is simply done by multiplying the quaternions
representing each rotation. The product of two quaternions $\q{q} =
(w_1,x_1,y_1,z_1)$ and $\q{p} = (w_2,x_2,y_2,z_2)$ is defined by the Hamilton
product:
\begin{equation}
	\q{q}\otimes\q{p} = \q{q}\q{p} = \matrix{
	w_1w_2 - x_1x_2 - y_1y_2 - z_1z_2\\
	w_1x_2 + x_1w_2 - y_1z_2 + z_1y_2\\
	w_1y_2 + x_1z_2 + y_1w_2 - z_1x_2\\
	w_1z_2 - x_1y_2 + y_1x_2 + z_1w_2
	}
\end{equation}
If \q{q_1},\q{q_2},\q{q_3},\ldots are the quaternions representing the
rotations then the composed rotation \q{q_{123\ldots}} is
\begin{equation}
	\q{q_{123\ldots}} = \ldots\q{q_3}\q{q_2}\q{q_1}
\end{equation}
It is worth noting that similarly to the composition of rotations, the
quaternion multiplication is not commutative. In the example above, we have
executed the rotation 1, followed by the rotation 2, followed by the rotation 3,
\ldots.

Applying a rotation \q{q} to a vector \v{v} is done by evaluating its
conjugation by \q{q}. The vector is considered as a quaternion with a null real
coordinate $\q{v} = (0,\v{v})$ (vector quaternion).
\begin{equation}
	\q{v'} = \matrix{0\\\v{v'}} = \q{q}\q{v}\q{q^{*}}
\end{equation}

The opposite operation, finding the quaternion that will move the origin vector $\v{v}_o$ onto the final vector
$\v{v}_{f}$ is done in two steps. First creating a quaternion \q{q_{of}} whose vector part is the vector product of the
two vectors and the real part is the sum of their norm and their scalar product.
\begin{equation}
	\q{q_{of}} = \matrix{||\v{v}_o|| + ||\v{v}_f|| + \v{v}_o\cdot\v{v}_f\\\v{v}_o\times\v{v}_f}
\end{equation}
Then the quaternion must be normalized to represent a rotation.
