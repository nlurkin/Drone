A proportional-integral-derivative (PID) controller is a feedback control loop mechanism. The aim is to reach a given
reference value (called set point) for the intput by regulating its output. The output $u(t)$ of the controller is the
sum of three terms that are proportional conrrespondingly to the error $e(t)$ (P term), the integral of the error (I
term) and the derivative of the error (D term). Each term is multiplied by a gain, $K_P$,$K_I$ and $K_D$ respectively,
that will give their relative importance and fix the properties of the controller: overshoot, reactivity, stability.
\begin{equation}
	u(t) = K_P e(t) + K_I\int_0^t{e(\tau)d\tau} + K_D \frac{de(t)}{dt}
\end{equation}
The P term can be interpreted as a reponse to a perturbation. It is subject to a steady state error that reduces when
$K_P$ rises. But increasing $K_P$ also enhances the oscillations of the system. Adding the I term removes the steady
state error as its action continue to affect the output as long as a stable state on the set point has not been
reached. The time to reach the set point is reduced when the gain of the integral term is increased but as for the
proportional term it also enhances the oscillations and overshoots. Finally the derivative term acts as a predictive
term by linear interpolation that help damping the oscillations

%TODO high pass and low pass filters for the derivative term (see astrom-ch6)

%TODO address integral windup
%TODO address derivative kick
%TODO feed forward when changing the setpoint
%TODO think about a cascaded PID : the first controls the torque, the second control the motor: 
% motor_response = PID(measured_torque, PID(sensor, setpoint));
%TODO look at the different forms of PID for optimisation

The controller described hereafter is able to drive the drone towards the desired attitude and linear velocity, the set
points for this application. The inputs of the controller are the measurements of angular rates, linear velocity and
attitude quaternion. The output are the torque and thrust that need to be applied on the model.  

This goal is reached by cascading two PID controllers. The first one is controlling the linear velocity of the model,
using it as an input. As the quadricopter is not able to produce a thrust in any direction but only along its z axis,
acting on the acceleration in any other direction can only be achieved by rotating the body frame z axis in the
inertial frame. The control on the acceleration is then realized by acting on the total thrust and the orientation of
the z axis in the inertial frame. The error $\v{v}_e$ is defined as the difference between the reference velocity
$\v{v}_{Ref}$ and the measured velocity $\v{v}_m$
\begin{equation}
	\v{v}_{err} = \v{v}_{Ref} - \v{v}_m
\end{equation}
%TODO adapt this with the real PID when implemented
 The total thrust in the inertial frame is proportional to this error but the effect of the gravity has to be taken into
 account and a constant term is added (feed-forward). Only the disturbance with respect to gravity are taken into account by this
 controller.
 \begin{equation}
 	\v{T}_{out} = \v{v}_{err} - \v{g}
 \end{equation}
This thrust and the quaternion representing the rotation needed to move the body z axis onto the thrust direction are
the output of this controller and are used as set point for the second.
