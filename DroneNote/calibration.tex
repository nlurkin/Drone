This section is describing the calibration procedure required to determine the values of the (diagonal) inertia matrix
and the four motor response reduced parameters for each motor.


Starting with \eq{torqueEquation}, introducing our new parametrisation from \eq{reducedTauParallel} and
\eq{reducedTauPerp} and explicitely introducing the time dependance.
\begin{equation}
	\v{\tau}(t) = \sum_i{\v{R}_{i,\tau} P_i^2(t)} = I\v{\dot{\omega}}(t) + \v{\omega}(t)\times(I\v{\omega}(t))  
\end{equation}

Performing two measurements at times $t_1$ and $t_2$ and taking the difference gives
\begin{equation}
\begin{split}
	\sum_i{\v{R}_{i,\tau}(P_i^2(t_1)-P_i^2(t_2))} = I(\v{\dot{\omega}}(t_1)-\v{\dot{\omega}}(t_2))\\
	+ (\v{\omega}(t_1)\times(I\v{\omega}(t_1))-\v{\omega}(t_2)\times(I\v{\omega}(t_2)))
\end{split}
\end{equation}
Or the developped form
%TODO check the signs
\begin{equation}
\begin{split}
	&\sum_i{(P_i^2(t_1)-P_i^2(t_2))\matrix{R_{i,\tau_x}\\ R_{i,\tau_y}\\ R_{i,\tau_z}}} =
	\\&I\matrix{ (\dot{\omega}_x(t_1) - \dot{\omega}_x(t_2)) +
	I_1(\omega_y(t_1)\omega_z(t_1) - \omega_y(t_2)\omega_z(t_2))\\
	(\dot{\omega}_y(t_1) - \dot{\omega}_y(t_2)) +
	I_2(\omega_x(t_1)\omega_z(t_1) - \omega_x(t_2)\omega_z(t_2))\\
	(\dot{\omega}_z(t_1) - \dot{\omega}_z(t_2)) +
	I_3(\omega_x(t_1)\omega_y(t_1) - \omega_x(t_2)\omega_y(t_2))
	}
\end{split}
\end{equation}

This expression is the basis of the calibration and its form suggest two things:
\begin{itemize}
  \item If the motor powers are not changing between the measurements, the expression becomes independant of the $R_i$
  and the $P_i$. This allows the determination of $I_1$, $I_2$ and $I_3$. This operation is basically applying a
  rotation to the body and observing how the different axis interacts with each other.
  \item If the power of only one motor is varying the contributions of all others are cancelled and the expression
  allows to determine the motor response (if the inertia matrix is known).
\end{itemize}

In the first case the reduced inertia parameters are
\begin{align}
	I1 = \frac{\dot{\omega}_x(t_2)-\dot{\omega}_x(t_1)}{\omega_y(t_1)\omega_z(t_1)-\omega_y(t_2)\omega_z(t_2)}\\
	I2 = \frac{\dot{\omega}_y(t_2)-\dot{\omega}_y(t_1)}{\omega_x(t_1)\omega_z(t_1)-\omega_x(t_2)\omega_z(t_2)}\\
	I3 = \frac{\dot{\omega}_z(t_2)-\dot{\omega}_z(t_1)}{\omega_x(t_1)\omega_y(t_1)-\omega_x(t_2)\omega_y(t_2)}
\end{align}

It gives the relative value of each component en the diagonal elements of the inertia matrix but not the absolute value
which is necessary for the appropriate control. However knowing the value of one of the element fixes the value of all
the others. Due to the symetry of the model, the z component is easier to measure than the others and can be fixed
experimentally by the rotation pendulum method\todo{ref}. If $I_{zz}$ is fixed then
\begin{align}
	I_{xx} = &I_{zz}\frac{1-I_2}{1-I_1I_2}\\
	I_{yy} = &I_{zz}\frac{1 - I_2 + I_3 - I_1I_2I_3}{1-I_1I_2}
\end{align}

In the second case the response parameters of the motor are
\begin{align}
	R_{i,\tau_x} = I_{xx}\frac{(\dot{\omega}_x(t_1) - \dot{\omega}_x(t_2)) +
	I_1(\omega_y(t_1)\omega_z(t_1) - \omega_y(t_2)\omega_z(t_2))}{P_i^2(t_1)-P_i^2(t_2)}\\
	R_{i,\tau_y} = I_{yy}\frac{(\dot{\omega}_y(t_1) - \dot{\omega}_y(t_2)) +
	I_2(\omega_x(t_1)\omega_z(t_1) - \omega_x(t_2)\omega_z(t_2))}{P_i^2(t_1)-P_i^2(t_2)}\\
	R_{i,\tau_z} = I_{zz}\frac{(\dot{\omega}_z(t_1) - \dot{\omega}_z(t_2)) +
	I_3(\omega_x(t_1)\omega_y(t_1) - \omega_x(t_2)\omega_y(t_2))}{P_i^2(t_1)-P_i^2(t_2)}
\end{align}

%TODO reste � faire la d�termination de R_T


